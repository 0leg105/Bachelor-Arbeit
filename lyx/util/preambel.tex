% Pakete einbinden, die benötigt werden
\usepackage{scrpage2}
\usepackage{graphicx}             % Bilder einbinden
\usepackage{xcolor}               % Color support
\usepackage{amsmath}              % Matheamtische Formeln
\usepackage{amsfonts}             % Mathematische Zeichensätze
\usepackage{amssymb}              % Mathematische Symbole
\usepackage{float}                % Fließende Objekte (Tabellen, Grafiken etc.)
\usepackage{booktabs}             % Korrekter Tabellensatz
\usepackage[printonlyused]{acronym}  % Abkürzungsverzeichnis [nur verwendete Abkürzugen]
\usepackage{makeidx}              % Sachregister
\usepackage{listings}             % Source Code listings
\usepackage{listingsutf8}         % Listings in UTF8
\usepackage[hang,font={sf,footnotesize},labelfont={footnotesize,bf}]{caption} % Beschriftungen
\usepackage[scaled]{helvet}       % Schrift Helvetia laden
\usepackage[sf,bf,small]{titlesec} % Einstellungen für Überschriften
\usepackage[absolute]{textpos}	  % Absolute Textpositionen (für Deckblatt)
\usepackage{calc}                 % Berechnung von Positionen
\usepackage{blindtext}            % Blindtexte
\usepackage[bottom=40mm,left=35mm,right=35mm,top=30mm]{geometry} % Ränder ändern
\usepackage{setspace}             % Abstände korrigieren
\usepackage{ifthen}               % Logische Bedingungen mit ifthenelse
\usepackage{scrhack}              % Get rid of tocbasic warnings
\usepackage{lastpage}			  % Anzahl Seiten; für Titelblattt
%\usepackage[pagebackref=false]{hyperref}  % Hyperlinks
%\usepackage{rotating}             % Seiten drehen

%\setlength{\bibitemsep}{1em}     % Abstand zwischen den Literaturangaben
%\setlength{\bibhang}{2em}        % Einzug nach jeweils erster Zeile

% Farben definieren
\definecolor{linkblue}{RGB}{0, 0, 100}
\definecolor{linkblack}{RGB}{0, 0, 0}
\definecolor{comment}{RGB}{63, 127, 95}
\definecolor{darkgreen}{RGB}{14, 144, 102}
\definecolor{darkblue}{RGB}{0,0,168}
\definecolor{darkred}{RGB}{128,0,0}
\definecolor{javadoccomment}{RGB}{0,0,240}

% Einstellungen für das Hyperlink-Paket
\hypersetup{
    colorlinks=true,      % Farbige links verwenden       
%    allcolors=linkblue,
    linktoc=all,          % Links im Inhaltsverzeichnis
    linkcolor=linkblack,  % Querverweise
    citecolor=linkblack,  % Literaturangaben
	filecolor=linkblack,  % Dateilinks
	urlcolor=linkblack    % URLs
}

% Einstellungen für Quelltexte
\lstset{     
      xleftmargin=0.2cm,     
      basicstyle=\footnotesize\ttfamily,
      keywordstyle=\color{darkgreen},
      identifierstyle=\color{darkblue},
      commentstyle=\color{comment}, 
      stringstyle=\color{darkred}, 
      tabsize=2,
      lineskip={2pt},
      columns=flexible,
      inputencoding=utf8,
      captionpos=b,
      breakautoindent=true,
	  breakindent=2em,
	  breaklines=true,
	  prebreak=,
	  postbreak=,
      numbers=none,
      numberstyle=\tiny,
      showspaces=false,      % Keine Leerzeichensymbole
      showtabs=false,        % Keine Tabsymbole
      showstringspaces=false,% Leerzeichen in Strings
      morecomment=[s][\color{javadoccomment}]{/**}{*/},
      literate={Ö}{{\"O}}1 {Ä}{{\"A}}1 {Ü}{{\"U}}1 {ß}{{\ss}}2 {ü}{{\"u}}1 {ä}{{\"a}}1 {ö}{{\"o}}1
}

\urlstyle{same}

\titlespacing{\paragraph}{0pt}{1ex}{2.0ex}
\titlespacing{\subsubsection}{0pt}{3ex}{0.0ex}
\titlespacing{\subsection}{0pt}{4ex}{0.2ex}
\titlespacing{\section}{0pt}{7ex}{1ex}
\titleformat*{\subsubsection}{\sffamily\itshape\bfseries\small}
\titleformat*{\paragraph}{\sffamily\bfseries\small}


% Einstellungen für Überschriften
\renewcommand*{\chapterformat}{%
  \Large\chapapp~\thechapter   % Große Schrift
  \vspace{0.3cm}               % Abstand zum Titel des Kapitels
}

% Abstände für die Überschriften setzen

% In der Kopfzeile nur die kurze Kapitelbezeichnung (ohne Kapitel davor)
\renewcommand*\chaptermarkformat{\thechapter\autodot\enskip}
\automark[chapter]{chapter}

% Einstellungen für Schriftarten
\setkomafont{pagehead}{\normalfont\sffamily}
\setkomafont{pagenumber}{\normalfont\sffamily}
\setkomafont{paragraph}{\sffamily\bfseries\small}
\setkomafont{subsubsection}{\sffamily\itshape\bfseries\small}
\addtokomafont{footnote}{\footnotesize}
\setkomafont{chapter}{\LARGE\selectfont\bfseries}

% Wichtige Abstände
\setlength{\parskip}{0.2cm}  % 2mm Abstand zwischen zwei Absätzen
\setlength{\parindent}{0mm}  % Absätze nicht einziehen
\clubpenalty = 10000         % Keine "Schusterjungen"
\widowpenalty = 10000        % Keine "Hurenkinder"
\displaywidowpenalty = 10000 % Keine "Hurenkinder"
\renewcommand{\footnotesize}{\fontsize{9}{10}\selectfont} % Größe der Fußnoten
\setlength{\footnotesep}{8pt} % Abstand zwischen den Fußnoten

% Index erzeugen
\makeindex

% Einfacher Font-Wechsel über dieses Makro
\newcommand{\changefont}[3]{
\fontfamily{#1} \fontseries{#2} \fontshape{#3} \selectfont}

% Eigenes Makro für Bilder
\newcommand{\bild}[3]{
\begin{figure}[h]
  \centering
  \includegraphics[width=#2]{#1}
  \caption{#3}
  \label{#1}
\end{figure}}

% Wo liegt Sourcecode?
\newcommand{\srcloc}{src/}

% Wo sind die Bilder?
\graphicspath{{bilder/}}

% Makros für typographisch korrekte Abkürzungen
\newcommand{\zb}[0]{z.\,B.\ }
\newcommand{\dahe}[0]{d.\,h.\ }
\newcommand{\ua}[0]{u.\,a.\ }

% Flags für Veröffentlichung und Sperrvermerk
\newboolean{hsmapublizieren}
\newboolean{hsmasperrvermerk}
