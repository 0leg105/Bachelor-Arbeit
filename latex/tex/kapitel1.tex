\chapter{Einleitung}

\section{Erster Abschnitt}

Einleitung zur Arbeit.

Möglicherweise noch einmal unterteilt in Unterabschnitte.

\subsection{Textauszeichnungen}
\label{Einleitung:Textauszeichnungen}
\index{Auszeichnungen!im Text}

Man kann Text auch \textit{kursiv} oder \textbf{fett} setzen. Es gibt Bindestrichte -, Gedankenstriche -- und lange Striche ---.


\subsection{Anführungszeichen}

Deutsche Anführungszeichen gehen so: "`dieser Text steht in \glq Anführungszeichen\grq; alles klar?"'.


\subsection{Abkürzungen}
\index{Abkürzungen}
\index{Abbreviation|see{Abkürzungen}}

Eine \ac{ABK} wird bei der ersten Verwendung ausgeschrieben\footnote{Ausschreiben bedeutet, dass man nicht die Abkürzung sondern die lange Form verwendet.}. Danach nicht mehr: \ac{ABK}. Man kann allerdings die Langform\footnote{\blindtext} explizit anfordern: \acl{ABK} oder die Kurzform \acs{ABK} oder auch noch einmal die Definition: \acf{ABK}.

Mehr dazu findet sich im Kapitel~\ref{Einleitung:Textauszeichnungen} auf Seite~\pageref{Einleitung:Textauszeichnungen}.


\subsubsection{Noch ein Unterabschnitt}

\paragraph{Eine Absatzüberschrift}
\blindtext[1]


\subsection{Literaturarbeit}

Wichtig ist das korrekte Zitieren von Quellen, wie es auch von \cite{Kornmeier2011} dargelegt wird. Interessant ist in diesem Zusammenhang auch der Artikel von \cite{Vixie2007}. Häufig werden die Zitate auch in Klammern gesetzt, wie bei \citep{Kornmeier2011}.

\blindtext[4]