% -------------------------------------------------------
% Abstrakt / Abstract
% Achtung: Wenn Sie im Abstrakt Anführungszeichen verwenden wollen, dann benutzen Sie
%          nicht "` und "', sondern \enquote{}. "` und "' werden nicht richtig
%          erkannt.

% Kurze (maximal halbseitige) Beschreibung, worum es in der Arbeit geht auf Deutsch
\newcommand{\hsmaabstractde}{Die Arbeit handelt von Entscheidungssystemen der Game-AI. Entscheidungssysteme dienen der Entscheidungsfindung von Non-Player Charactern (NPC) in Videospielen und dienen der Immersion. Dabei werden die Entscheidungssysteme Finite State Machine (FSM), Behavior Tree (BT) und Goal oriented Action Planning (GOAP) erklärt, miteinander verglichen und anschließend bewertet. Im Fokus liegen das Entscheidungssystem GOAP, zu dem im Vergleich zu anderen Entscheidungssystemen wenige Bibliotheken und Dokumentationen existieren. Eine Bewertung der drei Enscheidungssystemen wird dem Entwickler die Wahl dieser erleichtern. Für den Vergleich werden die Entscheidungssysteme in ein Videospiel implementiert. Aus der gewonnen Erfahrung, Benchmarks und wissenschaftlicher Literatur werden diese abschließend bewertet. Die FSM wird dabei für einfache, der BT für mittlere bis komplexe und GOAP für komplexe NPC-Verhalten empfohlen.}

% Kurze (maximal halbseitige) Beschreibung, worum es in der Arbeit geht auf Englisch
\newcommand{\hsmaabstracten}{The thesis focuses on decision-making systems in Game AI. These systems are responsible for the decision-making of Non-Player Characters (NPCs) in video games and contribute to immersion. The decision-making systems Finite State Machine (FSM), Behavior Tree (BT), and Goal-Oriented Action Planning (GOAP) are explained, compared, and evaluated. The primary focus is on GOAP, as there are fewer libraries and less documentation available for it compared to other decision-making systems. Evaluating these three systems will help developers make an informed choice. For the comparison, the decision-making systems are implemented in a video game. Based on the gained experience, benchmarks, and scientific literature, they are assessed in detail. The FSM is recommended for simple behaviors, BT for medium to complex behaviors, and GOAP for complex NPC behaviors.}
