% -------------------------------------------------------
% Abstrakt / Abstract
% Achtung: Wenn Sie im Abstrakt Anf\"{u}hrungszeichen verwenden wollen, dann benutzen Sie
%          nicht "` und "', sondern \enquote{}. "` und "' werden nicht richtig
%          erkannt.

% Kurze (maximal halbseitige) Beschreibung, worum es in der Arbeit geht auf Deutsch
\newcommand{\hsmaabstractde}{Die Arbeit handelt von Entscheidungssystemen der Game-AI. Entscheidungssysteme dienen der Entscheidungsfindung von Non-Player-Character (NPC) in Videospielen und ihrer Immersion. Hierbei werden die Entscheidungssysteme Finite State Machine (FSM), Behavior Tree (BT) und Goal Oriented Action Planning (GOAP) erkl\"{a}rt, miteinander verglichen und anschlie\ss{}end bewertet. Im Fokus liegt das Entscheidungssystem GOAP, für das im Vergleich zu anderen Entscheidungssystemen weniger Bibliotheken existieren. Eine Bewertung der drei Entscheidungssysteme wird dem Entwickler die Auswahl erleichtern. F\"{u}r den Vergleich werden die Entscheidungssysteme in ein Videospiel mithilfe der Godot Engine implementiert. Aus den gewonnenen Erfahrungen, Benchmarks und der wissenschaftlichen Literatur werden diese verglichen und abschlie\ss{}end bewertet. Die FSM wird f\"{u}r einfache, der BT f\"{u}r mittlere bis komplexe und GOAP f\"{u}r komplexe NPC-Verhalten empfohlen.}

% Kurze (maximal halbseitige) Beschreibung, worum es in der Arbeit geht auf Englisch
\newcommand{\hsmaabstracten}{The thesis deals with decision-making systems in game AI. Decision-making systems serve the purpose of making decisions for non-player characters (NPCs) in video games and enhancing their immersion. In this context, the decision-making systems Finite State Machine (FSM), Behavior Tree (BT), and Goal-Oriented Action Planning (GOAP) are explained, compared, and subsequently evaluated. The focus is on the GOAP system, for which fewer libraries exist compared to other decision-making systems. An evaluation of the three decision-making systems will facilitate the developer's selection. For comparison, the decision-making systems are implemented in a video game using the Godot Engine. Based on the gained experiences, benchmarks, and scientific literature, they are compared and finally evaluated. FSM is recommended for simple NPC behaviors, BT for medium to complex behaviors, and GOAP for complex NPC behaviors.}
