\chapter{L\"{o}sungskonzept}
\label{chap:l�sungskonzept}

Ziel dieser Thesis ist es, die Entscheidungssysteme Finite State Machine (FSM), Behavior Tree (BT) und Goal-Oriented Action Planning (GOAP) zu vergleichen und anschlie�end zu bewerten. Die \hyperref[chap:entscheidungssysteme]{Entscheidungssysteme} werden auf die folgenden Punkte verglichen: Erlernbarkeit, Implementation, Skalierbarkeit, Debugging, Performance und Speicherverbrauch.

F�r den Vergleich und die abschlie�ende Bewertung werden sowohl wissenschaftliche Literatur als auch eigene Erfahrungen herangezogen. Letztere basieren auf einer Implementierung der drei genannten Entscheidungssysteme in einem spezifischen Szenario. Aus diesem Szenario wird anschlie�end �ber Benchmarks die Performance und der Speicherverbrauch gemessen. Es soll insbesondere die Umsetzung des GOAP-Entscheidungssystem beschrieben werden. Die beiden Entscheidungssysteme FSM und BT wurden in den Grundkapitel\ref{} erl�utert.

\section{Implementierung}
\label{chap:lk implementierung}

Durch eine eigenst�ndige Implementation sollen praktische Erfahrungen in den Vergleich und die Bewertung der \hyperref[chap:entscheidungssysteme]{Entscheidungssysteme} einflie�en. Die Implementierung erfolgt in Form eines First-Person-Shooters (FPS)\ref{} Szenario unter Verwendung der \hyperref[chap:game engines]{Game-Engine} \hyperref[chap:godot]{Godot} 4.3. Die Spielwelt des Implementation ist dabei 2D-kartierbar und enth�lt Deckungen wie W�nde und S�ulen sowie eine vom Spieler gesteuerte Spielfigur und verschiedene NPC-Klassen. Der Spieler kann sich auf der Spielwelt frei bewegen und von NPCs erkannt werden. Diese NPCs sind dem Spieler feindlich gesinnt und besitzen einfache Aktionen und Ziele. Ihr Ziel ist es, den Spieler zu finden und zu eliminieren.

Um das Ziel zu erreichen, werden Aktionen in Form von \hyperref[chap:game-objects]{Komponenten} realisiert wie: schie�en, bewegen, umschauen, sehen, patrouillieren, Spieler-, Deckung suchen und nachladen. Die NPC-Klassen unterscheiden sich in ihren verwendeten Entscheidungssystemen, nutzen jedoch dieselben Komponenten zur Ausf�hrung ihrer Aktionen. Zwar werden f�r das Szenario eigenst�ndige Komponenten entwickelt, diese werden jedoch nicht weiter beschrieben.

\section{Benchmark}
\label{chap:benchmark}

Ein Benchmark ist ein standardisierter Test, der die Leistung von Systemen bewertet. Mit Hilfe der Resultate eines Benchmarks k�nnen Systeme verglichen werden. In der Softwareentwicklung werden Benchmarks h�ufig genutzt, um die Geschwindigkeit, Stabilit�t oder Ressourcennutzung von Programmen oder Ger�ten zu messen.

F�r den Vergleich wird die Performance und der Speicherverbrauch der drei \hyperref[chap:entscheidungssysteme]{Entscheidungssysteme} �ber die Benchmarks gemessen. W�hrend eines Benchmarks bewegt sich die Spielfigur in Drehungen auf zuf�llige Koordinaten der Spielwelt, woraufhin die NPCs entsprechend ihres Entscheidungssystem reagieren. Die Test wird unter identischen Bedingungen f�r alle Entscheidungssysteme durchgef�hrt, um eine faire Vergleichbarkeit zu gew�hrleisten. Die Ergebnisse werden in der Bewertung \ref{} anschlie�end analysiert und bewertet.

Der Performance-Bechnmark misst w�hrend der Ausf�hrung des Videospieles die Bilder-pro-Sekunde (FPS). Die FPS bezeichnen die Anzahl der Bilder, die pro Sekunde auf einem Bildschirm angezeigt werden. In Videospielen ist FPS eine wichtige Kennzahl f�r die Fl�ssigkeit und die visuelle Qualit�t einer Animation. Je mehr Bilder pro Sekunde angezeigt werden, desto fl�ssiger und somit ruckelfreier wirkt die Darstellung. Leistungsst�rkere Hardware kann besonders bei rechenintensiven Anwendungen h�here FPS erm�glichen. Ein akzeptables Ma� in Videospielen sind 30 FPS.

%Speicher-Test
Der Speicher-Test misst den Speicherverbrauch der drei \hyperref[chap:entscheidungssysteme]{Entscheidungssysteme} w�hrend der Laufzeit. Dabei wird der von den NPCs genutzte Arbeitsspeicher (RAM) in regelm��igen Intervallen in Bytes erfasst. Ziel ist es, Unterschiede im Speicherbedarf der einzelnen Entscheidungssysteme zu analysieren und ihre Effizienz zu bewerten. Dabei wird der von den NPCs genutzte Arbeitsspeicher (RAM) in regelm��igen Intervallen in Bytes erfasst.Je weniger Bytes verbraucht werden, desto effizienter ist das Entscheidungssystem in Bezug auf den Speicherverbrauch.

Ziel ist der Benchmarks ist es, Unterschiede im Speicherbedarf und Performance der einzelnen Entscheidungssysteme zu analysieren und ihre Effizienz zu bewerten.