\chapter{Zusammenfassung}
\label{chap:zusammenfassung}

In dieser Arbeit werden Game-AI Entscheidungssysteme hinsichtlich der Umsetzung und Funktionsweise untersucht. Zu den bekanntesten Entscheidungssystemen der Game-AI geh\"{o}ren die FSM der BT und der GOAP. Trotz erfolgreicher Spiele und Forderungen neuer Frameworks, die GOAP umsetzen, ist dieses Entscheidungssystem in der Game-AI zunehmend in den Hintergrund ger\"{u}ckt. Somit setzt die Arbeit bei der Untersuchung der Entscheidungssysteme den Fokus auf GOAP.

W\"{a}hrend FSM und BT als statische Ad-hoc-Behavioring-Methoden eine einfache Struktur aufweisen, ist GOAP als dynamisches Entscheidungssystem durch die Nutzung des A*-Algorithmus deutlich komplexer. Besonders die Funktionsweise von GOAP wird n\"{a}her betrachtet, wobei Suchalgorithmen wie der A*-Suchalgorithmus eine zentrale Rolle spielen.

Nach dem L\"{o}sungskonzept setzt die Arbeit die Implementierung der drei Entscheidungssysteme in einem Videospiel-Szenario des FPS-Genre in der Game-Engine Godot 4.3 um. Die Implementierung sieht vor aus den gewonnen Erfahrungen und weiteren Quellen die Entscheidungssysteme nach den folgenden Punkten zu vergleichen und abschlie\ss{}end zu bewerten: Erlernbarkeit, Umsetzung, Skalierbarkeit, Debugging, Performance und Speicherverbrauch. Bei der Implementierung des L\"{o}sungskonzept\ wird insbesondere die Umsetzung von GOAP beschrieben.

Die Entscheidungssysteme haben in den verglichenen Punkten ihre schw\"{a}chen und st\"{a}rken. Die Wahl des Entscheidungssystems h\"{a}ngt von den Anforderungen des Projekts ab. Je nach Projekt k\"{o}nnen die einzelnen Vergleichs-Punkte unterschiedlich gewichtet werden. F\"{u}r einfache NPC-Logiken eignet sich die FSM, w\"{a}hrend der BT f\"{u}r mittlere bis komplexe Verhaltensweisen besser geeignet ist. F\"{u}r komplexe NPC-Verhaltensweisen bietet GOAP die beste Skalierbarkeit, erfordert jedoch aufgrund seiner komplexen Funktionsweise und der schwierigen Implementierung mehr Aufwand.

Die Arbeit stellt eine Implementierung von GOAP vor, die Entwicklern einen Einblick in die Umsetzung des Entscheidungssystems erm\"{o}glicht. Abschlie\ss{}end erfolgt ein Vergleich der drei Entscheidungssysteme, gefolgt von einer Bewertung.