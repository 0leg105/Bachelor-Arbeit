\chapter{Typographie}

\section{Hervorhebungen}
\label{Einleitung:Textauszeichnungen}

Achten Sie bitte auf die grundlegenden Regeln der Typographie\index{Typographie}\footnote{Ein Ratgeber in allen Detailfragen ist \cite{Forssman2002}.}, wenn Sie Ihren Text schreiben. Hierzu gehören z.\,B. die Verwendung der richtigen "`Anführungszeichen"' und der Unterschied zwischen Binde- (-), Gedankenstrich (--) und langem Strich (---).

Wenn Sie Text hervorheben wollen, dann setzten Sie ihn \textit{kursiv} (Italic) und nicht \textbf{fett} (Bold). Fettdruck ist Überschriften vorbehalten; im Fließtext stört er den Lesefluss. Das \underline{Unterstreichen} von Fließtext ist im gesamten Dokument tabu und kann maximal bei Pseudo"=Code vorkommen.\index{Hervorhebungen}


\section{Anführungszeichen}

Deutsche Anführungszeichen gehen so: "`dieser Text steht in \glq Anführungszeichen\grq; alles klar?"'. Englische Anführungszeichen werden anders benutzt: ``this is an `English' quotation.''

\section{Abkürzungen}
\index{Abkürzungen}
\index{Abbreviation|see{Abkürzungen}}

Eine \ac{ABK} wird bei der ersten Verwendung ausgeschrieben. Danach nicht mehr: \ac{ABK}. Man kann allerdings die Langform explizit anfordern: \acl{ABK} oder die Kurzform \acs{ABK} oder auch noch einmal die Definition: \acf{ABK}.

Beachten Sie, dass bei Abkürzungen, die für zwei Wörter stehen, ein kleines Leerzeichen nach dem Punkt kommt: z.\,B. bzw. \zb{} und d.\,h. bzw. \dahe{}. Das Template bietet hierfür die beiden Makros \verb+\zb{}+ und \verb+\dahe{}+.


\section{Querverweise}

Querverweise auf eine Kapitelnummer macht man im Text mit \verb+\ref+ (Kapitel~\ref{Einleitung:Textauszeichnungen}) und auf eine bestimmte Seite mit \verb+\pageref+ (Seite~\pageref{Einleitung:Textauszeichnungen}). Man kann auch den Befehl \verb+\autoref+ benutzen, der automatisch die Art des referenzierten Elements bestimmt (\zb{} \autoref{Einleitung:Textauszeichnungen} oder \autoref{Kap2:Kopplungsformen}).


\section{Fußnoten}

Fußnoten werden einfach mit in den Text geschrieben und zwar genau an die Stelle\footnote{An der die Fußnote auftauchen soll.}


\section{Tabellen}

Tabellen werden normalerweise ohne vertikale Striche gesetzt, sondern die Spalten werden durch einen entsprechenden Abstand voneinander getrennt.\footnote{Siehe \cite[S. 89]{Willberg1999}.} Zum Einsatz kommen ausschließlich horizontale Linien (siehe Tabelle~\ref{Kap2:Kopplungsformen}).

\begin{table}[h]
  \caption{Ebenen der Kopplung und Beispiele für enge und lose Kopplung}
  \label{Kap2:Kopplungsformen}
  \renewcommand{\arraystretch}{1.2}
  \centering
  \sffamily
  \begin{footnotesize}
    \begin{tabular}{l l l}
    \toprule
    \textbf{Form der Kopplung} & \textbf{enge Kopplung} & \textbf{lose Kopplung}\\
    \midrule
    Physikalische Verbindung	&	Punkt-zu-Punkt	& 	über Vermittler\\
    Kommunikationsstil	&	synchron		&	asynchron\\
    Datenmodell	&	komplexe gemeinsame Typen	&	nur einfache gemeinsame Typen\\
    Bindung	&	statisch		&	dynamisch\\
    \bottomrule
    \end{tabular}
  \end{footnotesize}
  \rmfamily
\end{table}

Eine Tabelle fließt genauso, wie auch Bilder durch den Text. Siehe Tabelle~\ref{Kap2:Kopplungsformen}.

Manchmal möchte man Tabellen, in denen der Text in der Tabellenspalte umbricht. Hierzu dient die Umgebung \texttt{tabularx}, wobei \texttt{L} eine Spalte mit Flattersatz und \texttt{X} eine mit Blocksatz definiert. Die Breite der Tabelle kann über den Faktor vor \verb+\textwidth+ angegeben werden.

\begin{table}[h]
  \caption{Teildisziplinen der Informatik}
  \label{Kap2:Teildisziplinen}
  \renewcommand{\arraystretch}{1.2}
  \centering
  \sffamily
  \begin{footnotesize}
    \begin{tabularx}{0.9\textwidth}{l X L}
      \toprule
      \textbf{Gebiet} & \textbf{Definition} & \textbf{Beispiel}\\
      \midrule
      \emph{Praktische Informatik} & Informatik-Disziplinen, welche sich vorwiegend mit der Entwicklung und Anwendung der Software-Komponenten befassen & Programmentwicklung, Compilerbau; im Aufbau von z.B. Informationssystemen und Netzwerken ergeben sich Überlappungen mit der technischen Informatik \\
      \emph{Technische Informatik} & Informatik-Disziplinen, welche sich vorwiegend mit der Entwicklung und Anwendung der Hardware-Komponenten befassen & Digitaltechnik, Mikroprozessortechnik \\
      \emph{Theoretische Informatik} & Informatik-Disziplinen, welche sich mit der Entwicklung von Theorien und Modellen der Informatik befassen und dabei viel Substanz aus der Mathematik konsumieren & Relationenmodell, Objekt-Paradigmen, Komplexitätstheorie, Kalküle \\
      \emph{Angewandte Informatik} & Informatik als instrumentale Wissenschaft & Rechtsinformatik, Wirtschaftsinformatik, Geoinformatik \\
      \bottomrule
    \end{tabularx}
  \end{footnotesize}
  \rmfamily
\end{table}

\section{Harveyballs}

\begin{quote}
    Harvey Balls sind kreisförmige Ideogramme, die dazu dienen, qualitative Daten anschaulich zu machen. Sie werden in Vergleichstabellen verwendet, um anzuzeigen, inwieweit ein Untersuchungsobjekt sich mit definierten Vergleichskriterien deckt. \parencite{Wikipedia_HarveyBalls}
\end{quote}

\begin{table}[h]
  \caption{Beispiel für Harvey Balls}
  \label{tab:harveyexample}
  \centering
  \begin{tabular}{lccc}
    \toprule
    & Ansatz 1 & Ansatz 2 & Ansatz 3\\
    \midrule
    Eigenschaft 1	& \harveyBallNone & \harveyBallQuarter & \harveyBallHalf \\
    Eigenschaft 2	& \harveyBallHalf & \harveyBallThreeQuarter & \harveyBallFull \\
    Eigenschaft 3	& \harveyBallFull & \harveyBallThreeQuarter & \harveyBallQuarter\\
    \bottomrule
  \end{tabular}
\end{table}


\section{Aufzählungen}

Aufzählungen sind toll.

\begin{itemize}
  \item Ein wichtiger Punkt
  \item Noch ein wichtiger Punkt
  \item Ein Punkt mit Unterpunkten
    \begin{itemize}
      \item Unterpunkt 1
      \item Unterpunkt 2
    \end{itemize}
  \item Ein abschließender Punkt ohne Unterpunkte
\end{itemize}


Aufzählungen mit laufenden Nummern sind auch toll.

\begin{enumerate}
  \item Ein wichtiger Punkt
  \item Noch ein wichtiger Punkt
  \item Ein Punkt mit Unterpunkten
    \begin{enumerate}
      \item Unterpunkt 1
      \item Unterpunkt 2
    \end{enumerate}
  \item Ein abschließender Punkt ohne Unterpunkte
\end{enumerate}
