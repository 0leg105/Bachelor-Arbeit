\chapter{Vergleich}

%FSM
%So besitzt \textit{Half-Life (1998)} X Zust�nde, welche �ber eine HFSM �bersichtlicher w�ren.
 
%Ein wesentlicher Nachteil von FSM besteht darin, dass die Anzahl der �berg�nge exponentiell ansteigen kann, wenn die Anzahl der Zust�nde zunimmt. Dieses Problem l�sst sich in gewissem Ma�e durch den Einsatz von hierarchischen FSM (HFSM) reduzieren. Dennoch neigt eine FSM dazu, in bestimmten Situationen ungenau zu reagieren, beispielsweise wenn Zust�nde oder �berg�nge unvollst�ndig definiert sind. Au�erdem k�nnen Leistungsprobleme auftreten, insbesondere bei einer �berm��igen Anzahl von Zust�nden.

%Diese Einschr�nkungen f�hrten dazu, dass die Spieleindustrie nach alternativen Methoden f�r die Entscheidungsfindung von NPCs suchte. Ein Beispiel hierf�r ist GOAP von Jeff Orkin, welches einen neuen Ansatz durch einen Suchalgorithmus entwickelte.


%- With recent technological developments, FSMs, DTs and BTs for managing decisions of NPCs are deficient in some situations, such as bad performance on overgrown tree structures, repeating mistakes without learning, and selecting the same decisions without being adaptive to different conditions creating the need for more believable character management methods

%GOAP
\autocite{Schwab2021}:
 - GOAP system in not ideal though. First of all, it is resource-heavy thus it is impossible to reliably implement full-scale system on mobile platform (as of 2018 at least). Additionally, it is really hard to predict possible queue of actions which makes it hard to debug and a challenge to assign satisfying costs to every action. \autocite{Schwab2021}
 - FSM is most definitely not a viable alternative to GOAP. As stated in 1, its maintenance in ever-changing game development environment is too much of a hassle with complicated AI behavior. It is however the perfect solution for small projects because of it simplicity.
- BT is not perfect solution for more complicated AIs though. Developers need to spend a lot of time to create appropriate conditional nodes and link them with correct behaviors.

