\chapter{Videospiele}

Bei Videospielen handelt es sich um interaktive Medien, welche Spielern eine immersive und unterhaltsame Erfahrung bieten sollen. Sie erm�glichen eine Interaktion zwischen einem Spieler und einer Hardware, mit gelegentlich weiteren Spielern. Diese Interaktion erfolgt mithilfe von Eingabeger�ten in einer fiktionalen Spielwelt, welche �ber die Ausgabeger�te der Hardware visuell, akustisch oder haptisch simuliert wird. Bei den Eingabeger�ten kann es sich um eine Maus, Tastatur, Gamepad oder Touch-Display handeln. Der Spieler wird dabei durch seine Handlungen und seinen Konsequenzen innerhalb dieses Kontexts emotional beeinflusst.\autocite{Bergonse}



\section{Genres}

Ein Genre klassifiziert k�nstlerische Werke nach ihren Eigenschaften, wie Inhalt und Art. Wie auch Filme, Musik und Literatur geh�ren Videospiele auch zu k�nstlerischen Werken. Videospiele lassen sich in nach ihrer Spielart und Spielinhalte unterteilen. Klassische Videospiel-Genre w�ren: Abenteuer-, Shooter-, Rollen-, Strategie-, Horror- oder Simulations-Spiele. Ein Videospiel muss nicht unbedingt ein Genre erf�llen und kann in mehrere Genres eingegliedert werden. So f�llt das Videospiel �F.E.A.R First Encounter Assault Recon� in das Genre: Horror und Shooter.



\section{Perspektiven}

Wie auch in Filmen und Literatur erfasst der Nutzer sein Medium �ber eine Perspektive. So werden Filme �ber verschiedene Kameraperspektiven gedreht. Diese Kameraperspektiven sind auch in Videospielen vorhanden. In Videospielen spricht man von den Perspektiven: Third-Person und First-Person.

Bei der Third-Person-Perspektive handelt es sich um eine Perspektive au�erhalb der gesteuerten Spielfigur. Die Perspektive kommt in 2D und 3D-Videospielen vor. Sie kommt in allen Genres vor, sowohl auch Shootern.

Das Gegenst�ck zu Third-Person-Perspektive ist die First-Person-Perspektive, oder auch Egoperspektive genannt. Dabei handelt es sich um eine Perspektive aus den Augen der Spielfigur. Sie kommen in 3D-Videospielen und in Genre wie Shooter (FPS) vor.