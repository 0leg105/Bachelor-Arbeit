\chapter{Videospiele}

%1
%Videospiele sind interaktive Medien, die Spielern eine Unterhaltung und immersive Erfahrung bieten. Eine immersive Erfahrung ist, dass der Spieler durch seine Handlungen und Konsequenzen im Videospiel emotional beeinflusst wird. Die Interaktion im Videospiel erfolgt zwischen einem Spieler, einer Hardware und gelegentlich weiteren Spielern und mithilfe von Eingabeger�ten in einer fiktionalen Spielwelt. Bei den Eingabeger�ten kann es sich um eine Maus, Tastatur, Gamepad oder Touch-Display handeln. Die fiktionale Spielwelt wird �ber die Ausgabeger�te der Hardware visuell, akustisch oder haptisch simuliert. 

%2
%Bei Videospielen handelt es sich um interaktive Medien, welche Spielern eine immersive und unterhaltsame Erfahrung bieten sollen. Sie erm�glichen eine Interaktion zwischen einem Spieler und einer Hardware, mit gelegentlich weiteren Spielern. Diese Interaktion erfolgt mithilfe von Eingabeger�ten in einer fiktionalen Spielwelt, welche �ber die Ausgabeger�te der Hardware visuell, akustisch oder haptisch simuliert wird. Bei den Eingabeger�ten kann es sich um eine Maus, Tastatur, Gamepad oder Touch-Display handeln. Der Spieler wird dabei durch seine Handlungen und seinen Konsequenzen innerhalb dieses Kontexts emotional beeinflusst.\autocite{Bergonse}

%Bei Videospielen handelt es sich um interaktive Medien, die Spielern eine immersive und unterhaltsame Erfahrung bieten. Die Interaktion erfolgt �ber eine Hardware und gelegentlich weiteren Spielern. Diese Interaktion erfolgt mithilfe von Eingabeger�ten in einer fiktionalen Spielwelt. Bei den Eingabeger�ten kann es sich um eine Maus, Tastatur, Gamepad oder Touch-Display handeln. Die fiktionale Spielwelt wird �ber die Ausgabeger�te der Hardware visuell, akustisch oder haptisch simuliert. Innerhalb des Videospiels wird der Spieler durch seine Handlungen und Konsequenzen emotional beeinflusst. \autocite{Bergonse}

Bei Videospielen handelt es sich um interaktive Medien, die Spielern eine immersive und unterhaltsame Erfahrung bieten. Sie erm�glichen eine Interaktion zwischen einem Spieler, einer Hardware und gelegentlich weiteren Spielern. Diese Interaktion erfolgt mithilfe von Eingabeger�ten in einer fiktionalen Spielwelt. Bei den Eingabeger�ten kann es sich um eine Maus, Tastatur, Gamepad oder Touch-Display handeln. Die fiktionale Spielwelt wird �ber die Ausgabeger�te der Hardware visuell, akustisch oder haptisch simuliert. Innerhalb des Videospiels wird der Spieler durch seine Handlungen und Konsequenzen emotional beeinflusst. \autocite{Bergonse}

\section{Perspektiven}

%Nicht nur Filme und Literatur, sondern auch Videospiele werden von ihrem Nutzer �ber eine Perspektive erfasst. So werden Filme �ber verschiedene Kameraperspektiven gedreht, die auch in Videospielen vorhanden sind. In Videospielen spricht man von den folgenden beiden Perspektiven: Third-Person und First-Person.
%
%Die Third-Person-Perspektive ist eine Perspektive au�erhalb der gesteuerten Spielfigur und kommt in 2D und 3D Umgebung sowie in allen Genres vor.
%
%Das Gegenst�ck zu der Third-Person-Perspektive ist die First-Person-Perspektive, die auch als sogenannte Egoperspektive bezeichnet wird. Bei dieser handelt es sich um eine Perspektive aus den Augen der Spielfigur.


Nicht nur Filme und Literatur, sondern auch Videospiele werden von ihrem Nutzer �ber eine Perspektive erfasst. So werden Filme �ber verschiedene Kameraperspektiven gedreht, die auch in Videospielen vorhanden sind. In Videospielen spricht man von den folgenden Perspektiven: Third-Person und First-Person.

Die Third-Person-Perspektive ist eine Perspektive au�erhalb der gesteuerten Spielfigur und kommt in 2D und 3D Umgebung sowie in allen Genres vor.

Das Gegenst�ck zu der Third-Person-Perspektive ist die First-Person-Perspektive, die auch als Egoperspektive bezeichnet wird. Bei dieser handelt es sich um eine Perspektive aus den Augen der Spielfigur.

\section{Genres}

Genres kategorisieren k�nstlerische Werke nach ihren Eigenschaften, wie nach ihrem Inhalt und ihrer Art. Sowohl Filme, Musik und Literatur als auch Videospiele z�hlen zu k�nstlerischen Werken. So lassen sich Videospiele nach ihrem Spielinhalt und ihrer Spielart unterteilen. Klassische Videospiel-Genres sind Renn-, Rollen-, Strategiespiel, Horror und Shooter. Ein Videospiel muss nicht ausschlie�lich einem Genre angeh�ren, sondern kann auch mehreren Genres zugeordnet werden. So f�llt das Videospiel F.E.A.R First Encounter Assault Recon in die folgenden beiden Genres: Horror und Shooter.


%Ein Genre klassifiziert k�nstlerische Werke nach ihren Eigenschaften, wie Inhalt und Art. Wie auch Filme, Musik und Literatur geh�ren Videospiele zu k�nstlerischen Werken. Videospiele lassen sich in nach ihrer Spielart und Spielinhalte unterteilen. Klassische Videospiel-Genre sind Rennspiele, Shooter-, Rollenspiele, Strategie und Horror. Ein Videospiel muss nicht unbedingt ein Genre erf�llen und kann in mehrere Genres eingegliedert werden. So f�llt das Videospiel F.E.A.R First Encounter Assault Recon in das Genre: Horrorspiel und Shooter.


%\subsection{Subgenres}

%1
%K�nstlerische Werke werden durch Genres nach ihren Eigenschaften, wie nach ihrem Inhalt und ihrer Art, unterteilt.
%
%Videospiele werden nach ihren Spielinhalten und Spielarten durch Genres sowie nach ihren Darstellungen, Perspektiven und Spielmechaniken durch Subgenres kategorisiert. 
%
%durch Genres nach ihrem Inhalt und ihrer Art, sowie durch Subgenres
%
%Genres werden wiederum durch Subgenres nach den Darstellungen, Perspektiven und Spielmechaniken der Videospiele unterteilt. 
%
%Ein Genre kategorisiert sich wiederum nach Darstellung, Perspektiven und Spielmechanik

%2

%Videospiel-Genres werden wiederum durch Subgenres nach ihren Darstellungen, Perspektiven und Spielmechaniken unterteilt. 

Videospiel-Genres werden wiederum durch Subgenres nach ihren Darstellungen, Perspektiven und Spielmechaniken unterteilt. So k�nnen Videospiele des Genre Shooter entweder in der 2D oder 3D Umgebung spielen. Je nach Perspektive wird ein Shooter entweder als Third-Person-Shooter (TPS) oder als First-Person-Shooter (FPS) bezeichnet. Spielmechaniken beeinflussen die Spielweise des Videospielers. Unter anderem gibt es Spielmechaniken, die dem Spieler die M�glichkeit geben, K�mpfe durch Schleichen zu vermeiden. Shooter mit dieser Spielmechanik werden als Stealth-Shooter bezeichnet. Im Gegensatz zum Stealth-Shooter zwingt der Action-Shooter den Spieler aktiv zum Kampf, wie im Videospiel F.E.A.R.


%Ein Videospiel des Genre Shooter l�sst sich nach Darstellung, Perspektiven und Spielmechanik unterscheiden. Dabei gibt es Shooter, die entweder in 2D- oder 3D-Umgebungen spielen. Hinsichtlich der Perspektive wird zwischen Third-Person-Perspektive und First-Person-Perspektive unterschieden. Shooter aus der Third-Person-Perspektive werden als Third-Person-Shooter (TPS) und Shooter der First-Person-Perspektive als First-Person-Shooter (FPS) bezeichnet.
%
%Spielmechaniken beeinflussen die Spielweise des Videospielers. So gibt es Spielmechaniken welche dem Spieler die M�glichkeit geben K�mpfe durch schleichen zu vermeiden. Shooter mit dieser Spielmechanik werden als Stealth-Shooter bezeichnet. Dadurch l�sst sich ein Videospiel des Genre Shooter in ein weiteres Subgenre einteilen. Im Gegensatz zum Stealth-Shooter steht der Action-Shooter, zu dem beispielsweise F.E.A.R (2005) geh�rt.




