\chapter{Einleitung}
\label{chap:einleitung}

In diesem Kapitel werden die Motivation, das Ziel und Aufbau der Arbeit erl\"{a}utert.

\section{Motivation}
\label{chap:motivation}

Die Anzahl an Ver\"{o}ffentlichungen von Videospielen w\"{a}chst stetig \ref{}. Videospiele werden hinsichtlich ihrer dynamischen und ver\"{a}nderbaren Spielwelt immer prozeduraler und immersiver. Ein Entscheidungssystem kann die Immersivit\"{a}t der Spielewelt erweitern oder auch einschr\"{a}nken, sollte sie nicht darauf angepasst sein. Insbesondere die Agenten der Spielwelt, die Non-Player-Character (\gls{NPC}), tragen durch ihre Aktionen zur Immersion der Spielwelt bei. Diese Aktionen werden von einem Entscheidungssystem gew\"{a}hlt. Die gew\"{a}hlten Aktionen sollen m\"{o}glichst logisch in die Spielwelt passen und daher sinnvoll ausgew\"{a}hlt werden.

Spieleentwickler sollten sich, basierend auf den spezifischen Anforderungen ihres Projekts, f\"{u}r ein geeignetes Entscheidungssystem entscheiden und dieses richtig implementieren. Falsche Entscheidungen oder Implementierungen f\"{u}hren nicht nur zu Zeit-, sondern auch Geldverlust. Insbesondere f\"{u}r Indie-Entwickler, die meistens auf kosteng\"{u}nstige Game-Engines angewiesen sind\autocite{}. Eine korrekte Implementierung des Entscheidungssystems ist wichtig f\"{u}r den Projekterfolg.

Zu den bekannten Entscheidungssystemen der Game-AI geh\"{o}ren: Finite State Machine (FSM), Behavior Tree (BT) und Goal Oriented Action Planning (GOAP). Das Entscheidungssystem GOAP ist durch seine dynamische Entscheidungsfindung bekannt und soll sich dynamischen Spielen besser anpassen als andere Entscheidungssysteme. Doch Dokumentationen und Bibliotheken zu GOAP sind mager vorhanden und werden gefordert\autocite{}. Aus den genannten Gr\"{u}nden ist eine tiefgehende Untersuchung und Entwicklung von AI Entscheidungssystemen, wie GOAP relevant.

Durch \"{A}nderungen an der Bezahlstruktur unter der Unity Game-Engine, wenden sich viele Spieleentwickler nun vermehrt alternativen Engines wie Godot zu\ref{}. Godot ist eine kostenfreie Spiele-Engine und an Relevanz stetig zunimmt\autocite{}. Trotz der steigenden Popularit\"{a}t gibt es nur wenige wissenschaftliche Arbeiten, die sich mit der Implementierung eines Entscheidungssystems in Godot besch�ftigen. F\"{u}r Spieleentwickler w�re es daher ein Vorteil, mehr \"{u}ber die Implementierung eines solchen Systems in Godot zu erfahren. 

Entscheidungssysteme gewinnen neben der Spieleindustrie auch eine bedeutende Rolle in der Industrie. So sollen beispielsweise BTs in der Robotik Fine States Machines ersetzen k\"{o}nnen, welche bei zunehmender Komplexit�t anf�llige Verhaltensweisen erzeugen\autocite{}.

\section{Ziel}
\label{chap:ziel}

Ziel der Arbeit ist es eine theoretische Analyse und praktische Implementierung des GOAP-Systems durchzuf\"{u}hren. Dabei werden die Entscheidungssysteme FSM und BT als Konkurrenzsysteme einbezogen. Diese beiden Entscheidungssysteme werden erl�utert, verglichen und bewertet, wobei der Fokus auf GOAP liegt. Sie soll Entwickler auf GOAP aufmerksam machen und als Dokumentation dienen, die von Entwickler gefordert wird\autocite{}. 

Die theoretische Analyse st\"{u}tzt sich auf wissenschaftliche Literatur und bietet eine Einf\"{u}hrung in Entscheidungssysteme. Dabei werden die Grundlagen der Entscheidungssysteme erl�utert, der aktuelle Stand der Forschung dargestellt und insbesondere die Funktionsweise von GOAP beschrieben. Dies geschieht unter Ber\"{u}cksichtigung der Grundlagen von Suchproblemen und Suchalgorithmen.

Durch die praktische Implementierung der Entscheidungssysteme werden diese verglichen und bewertet. Die Implementierung erfolgt in einem einfachen First-Person-Shooter-Spiel \ref{} unter der Godot Game-Engine, in dem die Entscheidungssysteme auf verschiedene NPCs angewendet werden. Aus der Implementierung wird insbesondere die Umsetzung und Architektur von GOAP beschrieben.

F\"{u}r den Vergleich und die Bewertung werden insbesondere f\"{u}r Entwickler relevante Punkte ber\"{u}cksichtigt: Erlernbarkeit, Umsetzung, Skalierbarkeit, Debugging, Performance und Speicherverbrauch. Dabei flie\ss{}en unter anderem pers\"{o}nliche Erfahrungen, Erkenntnisse anderer Entwicklern und wissenschaftlicher Literatur ein.

Das Ergebnis der Arbeit ist ein praxisnaher Einblick in das Entscheidungssystem GOAP und die damit verbundenen Herausforderungen. Durch die Erl�uterung von FSM und BT erh�lt der Leser zudem eine breitere Perspektive auf Entscheidungssysteme und kann sich basierend auf den spezifischen Anforderungen eines Projekts f\"{u}r ein geeignetes System entscheiden.

\section{Aufbau der Arbeit}
\label{chap:aufbau der arbeit}

Die Arbeit ist in 10 Kapitel gegliedert. Nach diesem Einleitungskapitel beginnen mit Kapitel \ref{} die f\"{u}r diese Arbeit notwendigen Grundlagen. Die eigentliche Analyse und Implementierung der Entscheidungssysteme erfolgt, ab Kapitel \ref{}.

Zun�chst werden die Grundlagen von Videospielen in Kapitel \ref{} behandelt, wobei Videospiele, deren Genres und Perspektiven definiert werden. Anschlie\ss{}end folgt in Kapitel \ref{} eine Einf\"{u}hrung in die Entwicklung von Videospielen, die sich mit Entwicklungsumgebungen, sogenannten Game-Engines, befasst. Zudem wird eine kurze Einf\"{u}hrung in die Entwicklung von Spielelementen gegeben, darunter sogenannte Game-Objects. Abschlie\ss{}end wird der Bereich der Game-AI behandelt, der den Einsatz von AI in der Videospielentwicklung erl�utert. Kapitel \ref{} behandelt die Entscheidungsfindung von Agenten anhand verschiedener Entscheidungssysteme. Dazu geh\"{o}ren Ad-hoc Behaviour Authoring Methoden sowie Suchalgorithmen und deren Einsatz in der Robotik. Daraufhin werden in Kapitel \ref{} Suchprobleme und Suchalgorithmen erkl�rt, die f\"{u}r das Verst�ndnis von GOAP notwendig sind. Dazu geh\"{o}rt insbesondere der verwendete A*-Algorithmus. Die eigentliche Beschreibung von GOAP erfolgt in Kapitel \ref{}. Hier werden die Historie, Funktionsweise und ein Beispiel zur Veranschaulichung dargestellt. Zum Abschluss des Grundlagenkapitels wird in Kapitel \ref{} der State of the art im Videospielmarkt, der Game-Engines und der Entscheidungssysteme erl�utert.

Das L\"{o}sungskonzept wird in Kapitel \ref{} vorgestellt, gefolgt von Kapitel \ref{}, das die Implementierung des L\"{o}sungskonzepts beschreibt. Dabei liegt der Fokus auf der Umsetzung der GOAP-Architektur in Godot. In Kapitel \ref{} werden die Ergebnisse pr�sentiert, wobei die drei Entscheidungssysteme verglichen und bewertet werden. Das abschlie\ss{}ende Kapitel \ref{} fasst die Arbeit zusammen.
