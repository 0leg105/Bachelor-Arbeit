\definecolor{codegreen}{rgb}{0,0.6,0}
\definecolor{codegray}{rgb}{0.5,0.5,0.5}
\definecolor{codepurple}{rgb}{0.58,0,0.82}
\definecolor{backcolour}{rgb}{0.95,0.95,0.92}

\lstdefinelanguage{Godot}{
	keywords={class_name, func, for, return, in range, is, var, not, continue, or},
	morecomment=[l]{#}
}

\lstdefinelanguage{dict}{
	keywords={class_name, func, for, return, in range, is, var, not, continue},
}

\lstdefinestyle{mystyle}{
    backgroundcolor=\color{backcolour},   
    commentstyle=\color{codegreen},
    keywordstyle=\color{magenta},
    numberstyle=\tiny\color{codegray},
    stringstyle=\color{codepurple},
    basicstyle=\ttfamily\footnotesize,
    breakatwhitespace=false,         
    breaklines=true,                 
    captionpos=b,                    
    keepspaces=true,                 
    numbers=left,                    
    numbersep=5pt,                  
    showspaces=false,                
    showstringspaces=false,
    showtabs=false,                  
    tabsize=2
}

\lstset{style=mystyle}



\chapter{GOAP Umsetzung in Godot}

Die praktische Umsetzung des GOAP Systems wird nun in diesem Kapitel beschrieben. Die Umsetzung passiert auf Basis der Publikation von Jeff Orkin. Die Implementierung geschieht unter der Godot Engine 4.3. Das Kapitel wird zuerst die grundlegende Architektur von GOAP anhand eines Agenten beschreiben.[Kapitel weiter erläutern]



\section{GOAP Architektur}

Die folgende Abbildung soll die grundlegende Architektur von GOAP darstellen. Die Darstellung wird anhand eines Klassendiagramms in UML-Notation umgesetzt.

\begin{figure}[h]
  \centering
  \includegraphics[width=16cm]{GOAP_2/GOAP_UML}
	\captionsetup{justification=justified, format=plain}
  \caption{GOAP Architektur}
  \label{GOAP Architektur eines Agenten}
\end{figure}

Aus dem GOAP-Kapitel geht hervor, dass ein GOAP-System aus einem Planner, Zielen, Aktionen und einer FSM (Finite State Machine) besteht. Diese Komponenten erfüllen ihre jeweiligen Aufgaben, wie im Grundlagenkapitel über GOAP beschrieben.

Der GoapAgent bildet dabei die Hauptklasse und soll die Schnittstelle zur restlichen Spielwelt sein. Die Spielwelt kann ihm Informationen wie die Position des Spielers oder bestimmte Koordinaten innerhalb der Spielwelt übermitteln. Dabei erbt der GoapAgent von der Klasse Enemy. Die Klasse Enemy stellt Komponenten bereit, die es dem GoapAgent ermöglichen, mit der Spielwelt zu interagieren.

Der StateManager verwaltet die Zustände des NPC. Diese Zustände können über die Komponenten der Oberklasse Enemy verändert werden können.

Die FSM setzt die Aktionssequenz aus, welche im GoapAgent gespeichert wird. Sie kann neue Sequenzen an den GoapAgent anfordern.

Zur Generierung von Sequenzen und Festlegung des Zieles benutzt der GoapAgent die Klasse GoapPlanner. Der GoapPlanner besitzt dabei Objekte der Klasse GoapGoal. Aus diesen Objekten entscheidet sich der GoapPlanner für ein Zielzustand, zu welchem eine Aktionssequenz gesucht wird.
Der GoapPlanner sucht seine Sequenz mithilfe des A* Suchalgorithmus. Die Klasse AStarNode wird zur Erstellung von Knoten für den A* Suchalgorithmus benötigt. Ein AStarNode setzt dabei die Eigenschaften eines Suchbaum-Knoten um [siehe Kapitel Suchproblem, Knoten].

Die Klasse GoapAction definiert die Basisklasse für Aktionen. Eine Aktion repräsentiert dabei eine Kante im Suchbaum und wird entsprechend als solche im AStarNode-Objekt gespeichert.



\subsection{GoapAgent}

% Zusammenfassung
Folgende Abbildung stellt die Klasse GoapAgent dar. Die Darstellung wird anhand eines Klassendiagramms in UML-Notation umgesetzt. Der GoapAgent speichert folgende Klassenvariablen: goap\_planner, action\_sequence und den current\_step. Bei den Methoden handelt es sich um: update und follow\_sequence. 

% Klasse Enemy
Unter anderem erbt die Klasse Komponenten der Klasse Enemy. Die Komponenten dienen dem NPC zur Wahrnehmung und Manipulation der Videospielumgebung. Ein Beispiel einer solchen Komponente ist die vision\_component, welche über die update Methode aufgerufen, wird um den Spieler zu erfassen.

% Klassen-Attribute
Der goap\_planner ist eine Referenz auf die Klasse GoapPlanner, welche das Ziel und die action\_sequence des GoapAgent bestimmt. Das Array action\_sequence speichert Objekte vom Typ der Klasse GoapAction. Sie stellt die ermittelte Aktionssequenz durch den A* Algorithmus dar. Die Integer-Klassenvaraible current\_step handelt als Indexzeiger, welcher auf die aktuell auszuführende Aktion des action\_sequence verweist.

% Goap Agent
\begin{figure}[h]
  \centering
  \includegraphics[width=10cm]{GOAP_2/GoapAgent_UML}
	\captionsetup{justification=justified, format=plain}
  \caption{GOAP Agent}
  \label{GOAP Agent}
\end{figure}

% Methoden
Die Methode update wird durch eine \_process Funktion jedes \textit{Frame} aufgerufen, um kontinuierliche Logik wie Bewegungen oder Raycasting des GoapAgent zu verarbeiten. Die \_process Funktion wird von Godot 4.3 bereitgestellt. Sie befindet sich in der Komponente, welche das Objekt GoapAgent instanziiert. Die update Methode ruft Methoden der Klassen und Komponenten innerhalb des NPC auf. Der Parameter delta repräsentiert die Zeit seit dem letzten Frame und könnte an andere Methoden übergeben werden.

% update Methode
\lstinputlisting[firstline=2, language=Godot, linerange={12,14-17,19}, caption={update Methode des GoapAgent}, label=lst:caption]{code/goap_enemy.gd}

In der update Methode wird eine Abfrage gestellt, ob der goap\_planner eine neue action\_sequence generiert hat. Wurde eine neue action\_sequence generiert, so wird die Objektvariable current\_step auf $0$ zurückgesetzt und die neue action\_sequence aus dem goap\_planner abgerufen. Die action\_sequence wird dann durch die Methode follow\_sequence ausgeführt.

% follow_sequence Methode
Über die Methode follow\_sequence geschieht die Ausführung der Aktionen aus der action\_sequence. Sollte action\_sequence leer sein, bereits ausgeführt worden oder ungültig sein, so wird eine neue action\_sequence von dem goap\_planner gefordert. Ansonsten wird mithilfe des current\_step Index die jeweilige Aktion über ihre GoapAction.update Methode ausgeführt. Bei erfolgreicher Ausführung der Aktion wird der current\_step inkrementiert, um auf die nächste Aktion der action\_seqeunce zugreifen zu können.

\lstinputlisting[firstline=2, language=Godot, linerange={21-28}, caption={follow\_sequence Methode des GoapAgent}, label=lst:caption]{code/goap_enemy.gd}




\subsection{GoapPlanner}

% Zusammenfassung
Folgende Abbildung zeigt die Struktur des GoapPlanner, anhand eines Klassendiagramms in UML-Notation. Die Klasse besteht aus den Array Klassenvariablen actions, goals und current\_plan, den Dictionaries effect\_action\_table und current\_state, sowie dem Boolean create\_plan. Zu den Methoden gehören update, create_effect_action_dict, create_new_sequence, create_current_state_of_goals, create_path

\begin{figure}[h]
  \centering
  \includegraphics[width=10cm]{GOAP_2/GoapPlanner_UML}
	\captionsetup{justification=justified, format=plain}
  \caption{GoapPlanner}
  \label{GoapPlanner}
\end{figure}

Die GoapPlanner Methode update wird durch die update Methode des GoapAgent aufgerufen. Von update werden weitere Methoden und Abfragen durchgeführt. Einer der Methoden ist get\_best\_plan, welche das Ziel aus dem goal Array auswählt. Ändert sich das Ziel oder wird eine neue Sequence über create\_sequence Variable angefragt
Wird eine neue Sequenz angefordert oder ändert sich das Ziel während der Laufzeit, so soll eine neue Sequenz gesucht werden. Die Suche geschieht über die Methode \textit{create\_new\_sequence}. 

\lstinputlisting[firstline=2, language=Godot, linerange={16-22}, caption={\textit{update} Methode des GoapAgent}, label=lst:caption]{code/goap_a_star_planner.gd}

Über \textit{create\_new\_plan} werden Zustände des Zieles nach dem derzeitigen Zustand des Agenten überschrieben. Dieser Zustand wird als Zustand des Wurzelknoten dienen, mit dem die a\_star\_algorithm Methode nach der Sequenz sucht. Die Funktionsweise der a\_star\_algorithm Methode wird in der folgenden Abbildung kommentiert.

\lstinputlisting[firstline=2, language=Godot, linerange={42-59}, caption={A* Algorithmus des GoapPlanner}, label=lst:caption]{code/goap_a_star_planner.gd}

Man beachte, dass Godot 4.3 keine PriorityQueue besitzt und man diese selbst umsetzen müsste. Eine PriorityQueue speichert Knoten in sortierter Reihenfolge nach ihren $f(n)$ Kosten, sodass Knoten mit den niedrigsten Kosten bevorzugt abgerufen werden. Die Umsetzung befindet sich im Anhang...

Die expand\_node Methode fügt Kindknoten des expandierten Knoten in die open\_list, welche vorher von A$^*$ gewählt wurde. Es folgt die Instanziierung der Kosten $g(n)$, $h(n)$ und $f(n)$ der Kindknoten und die Hinzufügung in die \textit{open\_list}.

\lstinputlisting[firstline=2, language=Godot, linerange={61-79}, caption={Expandierungs Methode des GoapPlanner}, label=lst:caption]{code/goap_a_star_planner.gd}

Die get\_child\_nodes Methode sucht nach Kanten (Aktionen) welche die benötigten Zustände des Knoten erfüllen können. Dabei werden die benötigten Zustände mit den Zuständen der effect\_action\_dict verglichen.

\begin{lstlisting}[language=dict, caption={effect\_action\_dict aus der Implementierung}]
effect_action_dict = {
    "player_eliminated": [
        RangedAttackFromCover:<Node#82829117475>,
        MeleeAttack:<Node#82845894692>,
        RangedAttack:<Node#82812340258>
    ],
    "player_block_visited": [
        GoToNode:<Node#82862671909>
    ],
    "at_patrol_node": [
        GoToNode:<Node#82862671909>
    ],
    "at_cover_node": [
        GoToNode:<Node#82862671909>
    ],
    "bullets": [
        Reload:<Node#82879449126>
    ]
}
\end{lstlisting}

Das Dictionary speichert alle Zustände als \textit{key} und die Aktionen, welche den Zustand beeinflussen können als \textit{values}. So erhält man eine schnellere Zugriffszeit, als wenn man jeden Effekt einer Aktion im Array mit dem benötigten Zustand vergleicht. 

Wird eine Aktion gefunden, welche den Zustand erfüllt, so wird untersucht ob der Effekt, der Aktion den Zustand umsetzen kann. Erfüllt der Effekt den gewünschten Zustand, so wird ein Kindknoten erstellt. Dieser Kindknoten speichert die Kante (Aktion) die zu dem Knoten geführt hat, sowie den Effekt auf den current\_state. Die restlichen Inhalte des Knoten werden in der zuvor beschriebenen expand\_node Methode instanziiert.

\lstinputlisting[firstline=2, language=Godot, linerange={90-107}, caption={Methode zur Suche nach möglichen Kanten}, label=lst:caption]{code/goap_a_star_planner.gd}

Wenn A$^*$ einen Knoten erweitert, der keine zu erfüllenden Zustände mehr besitzt, wurde der optimale Pfad gefunden. Um die korrekte Aktionssequenz zu erhalten, müssen die Aktionen der Knoten rekursiv vom Zielknoten bis zum Startknoten zurückverfolgt werden. Dies wird mithilfe der \textit{create\_path} Methode durchgeführt.



\subsection{AStarNode}

...

\begin{figure}[h]
  \centering
  \includegraphics[width=10cm]{GOAP_2/AStarNode_UML}
	\captionsetup{justification=justified, format=plain}
  \caption{AStarNode}
  \label{AStarNode}
\end{figure}

Der AStarNode geht nach einem Knoten eine Suchbaums [siehe Kapitel: Knoten eines Suchbaums]. Er speichert die bis dahin erfüllten Zustände der Elternknoten im Attribut current\_state\_of\_goals, sowie alle Zielzustände die bis zu dem Knoten benötigt wurden im Attribut goal\_state. Die Objektvariable parent\_node speichert den Elternknoten, um später rekursiv auf die Kanten zurückschließen zu können. Die Kanten die zum Knoten geführt haben werden als GoapAction Objekt in der Objektvariable action gespeichert. Durch den Godot Konstruktor \_init wird der Knoten mit den eben genannten Attributen instanziiert.

Die Kosten $f(n)$ werden unter f\_cost und $g(n)$ -Kosten unter g\_cost gespeichert. Die Berechnung der Kosten passiert durch die GoapPlanner Methode expand\_node und werden über die setter- Methoden des AStarNode set\_g\_cost und set\_f\_cost initialisiert. Die Heuristik Kosten werden in der expand\_node Methode des GoapPlanner berechnet. Über die Methode get\_unsatasfied\_states werden alle Zustände zurückgegeben, welche noch nicht erreicht wurden. Die Größe des Arrays welches von get\_unsatasfied\_states gegeben wird repräsentiert die heuristischen Kosten der Aktion.

Wird eine Kante gewählt, so muss ein neuer Zustand für den dazugehörigen Knoten durch die apply\_action\_to\_state Methode generiert werden. Die Methode wird vom GoapPlanner über die Methode get\_child\_nodes auf dem zu expandierten Knoten aufgerufen. Dabei wird die Aktion als Kante und der derzeitige NPC-Zustand als Parameter übergeben. Basierend auf den Parameter initialisiert die Methode, die Attribute goal\_states und current\_state\_of\_goals für den neuen Knoten. Außerdem initialisiert sie den zu expandierten Knoten als parent\_node sowie die Aktion als action.

Die Prüfung ob die Zustände eines expandierten Knoten erfüllt wurden, wird durch is\_satasfied geprüft. Dabei schaut die Methode ob die Größe des Array welches über die Methode get\_unsatasfied\_states leer ist. Sollte das Array leer sein, so gibt es keine zu erfüllenden Zustände und der AStarNode ist der Zielknoten. Die getter-Methoden geben die Variablen des AStarNode zurück.




\subsection{GoapGoal}

Die Klasse \textit{GoapGoal} wird in der folgenden Abbildung mittels UML-Notation dargestellt. 

\begin{figure}[h]
  \centering
  \includegraphics[width=10cm]{GOAP_2/GoapGoal_UML}
	\captionsetup{justification=justified, format=plain}
  \caption{GoapGoal}
  \label{GoapGoal}
\end{figure}

Die Informationen zur Priorität und Gültigkeit werden aus der Objektvariable state\_manager gelesen. Die get\_best\_goal des GoapPlanner benötigt die Gültigkeit und Priorität, um in folge dessen das Ziel auszuwählen. Die Abfragen geschehen durch die Methoden get\_priority und is\_valid. Ein Ziel wird erst dann berücksichtigt, wenn es durch die Methode is\_valid als gültig bestätigt wurde. Anschließend kann dessen Priorität mithilfe von get\_priority ermittelt werden. 

Über die Methode get\_desired\_state werden die Zielzustände aufgerufen, welche der GoapPlanner zur Suche der Sequenz benötigt. 

Die Methode get\_goal\_name gibt den Klassennamen des GoapGoal zurück, welche von Debug Methoden genutzt werden kann.



\subsection{GoapAction}

Abbildung [Nummer] stellt die GoapAction Klasse nach UML-Notation dar.

\begin{figure}[h]
  \centering
  \includegraphics[width=10cm]{GOAP_2/GoapAction_UML}
	\captionsetup{justification=justified, format=plain}
  \caption{GoapAction}
  \label{GoapAction}
\end{figure}

Mit hilfe der Objektvariablen character\_body und state\_manager werden die Kosten $g(n)$ und die Gültigkeit der Aktion gelesen. Die Kosten $g(n)$ werden dabei über die get\_cost Methode berechnet. Für die Rückgabe der Gültigkeit ist die is\_valid Methode zuständig. Die is\_valid Methode wird dabei für die Suche nach der Aktion genutzt, sowie während der Ausführung durch die FSM. Der GoapPlanner wählt nur Aktionen, welche im derzeitigen Zustand des NPC auch durchführbar sind. Auch die FSM prüft die Gültigkeit der Aktionen, da diese zu späterer Zeit ausgeführt werden können und sich der Zustand wieder ändern kann und somit auch die Gültigkeit.

Eine get\_preconditions Methode gibt die vorausgesetzten Zustände zurück, welche von anderen Aktionen erfüllt werden können. Ob eine Aktion einen Zustand erfüllen kann, hängt von der get\_effects Methode ab, welche ein \textit{Dictionary} mit dem Zustand und dessen Wert zurückgibt. Stimmt der Wert des Zustands mit dem Zielzustand überein, dann erfüllt die Aktion den Zustand.

Die update Methode leitet die eigentliche Ausführung der Aktion ein und wird von der FSM gestartet. Die Hoffnung dabei ist, dass die Aktion den erwünschten Wert des Effekts umsetzt. Aufgrund der nicht-deterministischen Natur der Spielwelt kann es jedoch vorkommen, dass der angestrebte Effekt nicht erreicht wird. Wird der Effekt nicht erreicht, so wird eine neue Sequenz angefordert oder das Ziel ändert sich bis dahin. Die Ausführung der Aktion wird dabei über Komponenten der Klasse Npc umgesetzt. Die Methode get\_action\_name gibt den Klassennamen der GoapAction zurück.

%\lstinputlisting[firstline=2, language=Godot, linerange={}, caption={Die Aktion RangedAttack erbt von GoapAction}, label=lst:caption]{code/goap_ranged_attack.gd}

