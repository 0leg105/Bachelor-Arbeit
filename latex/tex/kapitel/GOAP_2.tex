\definecolor{codegreen}{rgb}{0,0.6,0}
\definecolor{codegray}{rgb}{0.5,0.5,0.5}
\definecolor{codepurple}{rgb}{0.58,0,0.82}
\definecolor{backcolour}{rgb}{0.95,0.95,0.92}

\lstdefinelanguage{Godot}{
	keywords={class_name, func, for, return, in range, is, var, not, continue, or},
	morecomment=[l]{#}
}

\lstdefinelanguage{Pseudo}{
	keywords={FUNCTION, IMPORT, END if, then, void},
	morecomment=[l]{//}
}

\lstdefinelanguage{dict}{
	keywords={class_name, func, for, return, in range, is, var, not, continue},
}

\lstdefinestyle{mystyle}{
    backgroundcolor=\color{backcolour},   
    commentstyle=\color{codegreen},
    keywordstyle=\color{magenta},
    numberstyle=\tiny\color{codegray},
    stringstyle=\color{codepurple},
    basicstyle=\ttfamily\footnotesize,
    breakatwhitespace=false,         
    breaklines=true,                 
    captionpos=b,                    
    keepspaces=true,                 
    numbers=left,                    
    numbersep=5pt,                  
    showspaces=false,                
    showstringspaces=false,
    showtabs=false,                  
    tabsize=2
}

\lstset{style=mystyle}



\chapter{GOAP Umsetzung in Godot}

Die praktische Umsetzung des GOAP Systems wird nun in diesem Kapitel beschrieben. Sie wird in Godot 4.3 umgesetzt und Hierbei wird insbesondere das Szenario vorgestellt. Codeabschnitte sind dem GIT zu entnehmen.

